\documentclass[12pt]{article}

\usepackage[german]{babel}
\usepackage{amsmath}
\usepackage{amssymb} % to display symbols for real numbers, integers etc. Usage: \mathbb{R}
\usepackage{graphicx}
\usepackage{listings} % to display programming code
%\usepackage[ngerman]{babel}
\usepackage{color}
\usepackage{relsize} % to display scaled math symbols (big summation symbol etc.)
\usepackage{textcomp}

\DeclareGraphicsExtensions{.pdf,.jpeg,.png}
\definecolor{listinggray}{gray}{0.9}
\definecolor{lbcolor}{rgb}{0.9,0.9,0.9}
\lstset{ % to display programming code in nice colors
	backgroundcolor=\color{lbcolor},
	tabsize=4,
	rulecolor=,
	language=matlab,
		basicstyle=\scriptsize, %for extra small font size
        upquote=true,
        aboveskip={1.5\baselineskip},
        columns=fixed,
        showstringspaces=false,
        extendedchars=true,
        breaklines=true,
        prebreak = \raisebox{0ex}[0ex][0ex]{\ensuremath{\hookleftarrow}},
		frame=single, %draw frame
        showtabs=false,
        showspaces=false,
        showstringspaces=false,
        identifierstyle=\ttfamily,
        keywordstyle=\color[rgb]{0,0,1},
        commentstyle=\color[rgb]{0.133,0.545,0.133},
        stringstyle=\color[rgb]{0.627,0.126,0.941},
        numbers=left,
        stepnumber=1,
        firstnumber=1,
        numberfirstline=true,
        linewidth=14cm,
}

\title{\"Ubungsblatt 10\\ \glqq K\"unstliche Intelligenz\grqq}
\author{T. Glei\ss ner, N. Lehmann, A. Zubarev}
\date{03.07.2015}
\begin{document}
\maketitle
%\renewcommand{\contentsname}{Table of Contents}
\tableofcontents
\newpage

\section{Parser / DCG}

\begin{lstlisting}[language=Prolog]
test(X) :- expr(X,"1.3*23+(-4)*(8+5)",[]).
test2(X) :- expr(X,"1*2",[]). /* extra Test */
 
expr(add(Y,Z)) --> term(Y), "+", expr(Z).
expr(X)     --> term(X).

term(mul(Y,Z)) --> n(Y), "*", n(Z).
term(mul(Y,Z)) --> n(Y), "*", term(Z).

term(X)     --> n(X).
term(X)     --> "(", expr(X), ")".

/******************** LÖSUNG **********************/
term(mul(X,Y)) --> n(X), "*", term(Y).
term(mul(X,Y)) --> "(", expr(X), ")", "*", term(Y).
/**************************************************/
 
n(X)        --> z(X,_).
n(X)        --> z(Y,_), ".", z(Z,A), {X is Y+10^(-A)*Z}.
n(X)        --> "-", z(Y,_), {X is -Y}.
n(X)        --> "-", z(Y,_), ".", z(Z,A), {X is -(Y+10^(-A)*Z)}.
% z(wert,anzahl der ziffern)
z(X,B)      --> num(Y), z(Z,A), {X is Y*10^A+Z, B is A+1}.
z(X,1)      --> num(X).
num(1)      --> "1".
num(2)      --> "2".
num(3)      --> "3".
num(4)      --> "4".
num(5)      --> "5".
num(6)      --> "6".
num(7)      --> "7".
num(8)      --> "8".
num(9)      --> "9".
num(0)      --> "0".
\end{lstlisting}

\newpage

\section{Certainty Factors / MYCIN Algebra}

\begin{lstlisting}[language=Prolog]
main :greeting, repeat, write('> '), read(X), do(X), X == quit,!.

algebra1(RuleCF,CF,AdjustedCF):X is RuleCF*CF/100, int_round(X,AdjustedCF).
algebra2(CF,OldCF,NewCF):minimum(CF,OldCF,NewCF).
algebra3(CF,RestCF,NewCF):CF >= 0, RestCF >= 0, X is CF + RestCF * (100 CF) / 100, int_round(X,NewCF).
algebra3(CF,RestCF,NewCF):CF < 0, RestCF < 0, X is (CF + RestCF * (100 + CF) / 100), int_round(X,NewCF).
algebra3(CF,RestCF,NewCF):(CF < 0; RestCF < 0), (CF > 0; RestCF > 0), abs_minimum(CF,RestCF,MCF), X is 100 * (CF + RestCF) / (100 MCF), int_round(X,NewCF).
int_round(X,I):X >= 0, I is integer(X+0.5).
int_round(X,I):X < 0, I is integer(X0.5).
minimum(X,Y,X) :X =< Y,!.
minimum(X,Y,Y) :Y =< X.
abs_minimum(A,B,X) :absolute(A, AA),absolute(B, BB),minimum(AA,BB,X).
absolute(X, X) :X >= 0.
absolute(X, Y) :X < 0, Y is X.
\end{lstlisting}

\end{document}