\documentclass[12pt]{article}

\usepackage[german]{babel}
\usepackage{amsmath}
\usepackage{amssymb} % to display symbols for real numbers, integers etc. Usage: \mathbb{R}
\usepackage{graphicx}
\usepackage{listings} % to display programming code
%\usepackage[ngerman]{babel}
\usepackage{color}
\usepackage{relsize} % to display scaled math symbols (big summation symbol etc.)
\usepackage{textcomp}

\DeclareGraphicsExtensions{.pdf,.jpeg,.png}
\definecolor{listinggray}{gray}{0.9}
\definecolor{lbcolor}{rgb}{0.9,0.9,0.9}
\lstset{ % to display programming code in nice colors
	backgroundcolor=\color{lbcolor},
	tabsize=4,
	rulecolor=,
	language=matlab,
		basicstyle=\scriptsize, %for extra small font size
        upquote=true,
        aboveskip={1.5\baselineskip},
        columns=fixed,
        showstringspaces=false,
        extendedchars=true,
        breaklines=true,
        prebreak = \raisebox{0ex}[0ex][0ex]{\ensuremath{\hookleftarrow}},
		frame=single, %draw frame
        showtabs=false,
        showspaces=false,
        showstringspaces=false,
        identifierstyle=\ttfamily,
        keywordstyle=\color[rgb]{0,0,1},
        commentstyle=\color[rgb]{0.133,0.545,0.133},
        stringstyle=\color[rgb]{0.627,0.126,0.941},
        numbers=left,
        stepnumber=1,
        firstnumber=1,
        numberfirstline=true,
        linewidth=14cm,
}

\title{\"Ubungsblatt 11\\ \glqq K\"unstliche Intelligenz\grqq}
\author{T. Glei\ss ner, N. Lehmann, A. Zubarev}
\date{10.07.2015}
\begin{document}
\maketitle
%\renewcommand{\contentsname}{Table of Contents}
\tableofcontents
\newpage

\section{Forward Chaining}

\begin{enumerate}
\item If X chirps and sings then X is a canary.
\item If X is a canary then X is yellow.
\item Tweety eats flies.
\item Tweety chirps.
\item Tweety sings.
\item Tweety is yellow.
\end{enumerate}

\subsection{Teilaufgabe a)}

Substituiere alle X mit Tweety, für die die Aussage wahr wird.\\
\\
If Tweety chirps (4) and sings (5) then Tweety is a canary.

\subsection{Teilaufgabe b)}

1) substituiere X durch Tweety\\
2) if Tweety chirps and sings $\Rightarrow$ if Tweety chirps $\wedge$ if Tweety sings\\
3) Tweety chirps (4) $\Rightarrow$ true\\
4) Tweety sings  (5) $\Rightarrow$ true\\
5) 3) $\wedge$ 4) $\Rightarrow$ then Tweety is a canary.

\newpage

\subsection{Teilaufgabe c)}

\textbf{Forward Chaining}
\begin{itemize}
\item dynamisches, datengesteuertes Verfahren
\item generiert neues Faktenwissen (auch unnötiges)
\item gut geeignet für unsichere Ziele und Auswahlmöglichkeiten
\end{itemize}
\textbf{Backward Chaining}
\begin{itemize}
\item statisches, zielgesteuertes Verfahren
\item Beweisrichtung: vom Ziel zu den Fakten
\item gut geeignet wenn alle Ziele und Auswahlmöglichkeiten bekannt sind
\end{itemize}

\section{Produktionssystem}

\begin{lstlisting}[language=Prolog]
initial_data([goal(select_budget),
	not_end_yet,
	legal_budgets([
		high_end,
		mid_end,
		low_end]),
	cpu(low_end, [i3]),
	cpu(mid_end, [i3, i5]),
	cpu(high_end, [i3, i5, i7]),
	gpu(low_end, [gpu_onboard]),
	gpu(mid_end, [gpu_onboard, gpu_extern]),
	gpu(high_end, [gpu_onboard, gpu_extern, gpu_water_cooled]),
	harddrive(low_end, [hdd]),
	harddrive(mid_end, [hdd, ssd]),
	harddrive(high_end, [hdd, ssd]).

% Select your budget
rule 100:
	[1: goal(select_budget),2: legal_budgets(LT)] ==>
	[retract(1), nl, write('Select a budget:'),nl,nl,
	write('Available budgets are:'),nl,write(LT),nl,nl,
	assert(goal(read_budget))].

rule 101:
	[1: goal(read_budget),2: legal_budgets(LB)] ==>
	[prompt('budget> ', B),	member(B,LB), retract(1),
	assert(budget(B)), assert(goal(assemble_cpu))].

rule 102:
	[1: goal(read_budget),2: legal_budgets(LT)] ==>
	[nl,write('Unknown. Select one of these:'),nl,
	write(LT)].

% CPU
rule 200:
	[1: goal(assemble_cpu),2: budget(B),cpu(B,CPU)] ==>
	[retract(1), nl, write('Select cpu. Your budget '),
	write(B), write(' allows for the following parts:'),nl,
	write(CPU),nl,nl,assert(goal(pick_cpu))].

rule 201:
	[1: goal(pick_cpu),2: budget(B),cpu(B, C)] ==>
	[prompt('cpu> ', CPU),member(CPU,C),retract(1),
	assert(picked_cpu(CPU)),assert(goal(assemble_gpu))].

rule 202:
	[1: goal(pick_cpu),2: budget(B),cpu(B,CPU)]	==>
	[write('Your cpu seems to not fit your budget. Choose from the following:'),nl,
	write(CPU),nl].

%GPU
rule 200:
	[1: goal(assemble_gpu),2: budget(B),gpu(B,GPU)] ==>
	[retract(1),nl,	write('Pick your gpu. Your budget '),
	write(B),write(' allows for the following parts:'),nl,
	write(GPU),nl,nl,assert(goal(pick_gpu))].

rule 201:
	[1: goal(pick_gpu),2: budget(B),gpu(B, G)] ==>
	[prompt('gpu> ', GPU),member(GPU,G),retract(1),
	assert(picked_gpu(GPU)),assert(goal(assemble_harddrive))].

rule 202:
	[1: goal(pick_gpu),2: budget(B), gpu(B,GPU)] ==>
	[nl,write('Your gpu seems to not fit your budget. Choose from the following:'),nl,
	write(GPU),nl].

% Hardrive
rule 300:
	[1: goal(assemble_harddrive),2: budget(B),harddrive(B,HDD)] ==>
	[retract(1),nl, write('Pick your harddrive. Your budget '),
	write(B), write(' allows for the following parts:'),nl,
	write(HDD),nl,nl, assert(goal(pick_harddrive))].

rule 301:
	[1: goal(pick_harddrive),2: budget(B),harddrive(B, G)] ==>
	[prompt('harddrive> ', HDD), member(HDD,G), retract(1),
	assert(picked_harddrive(HDD)), assert(goal(show_pc))].

rule 302:
	[1: goal(pick_harddrive),2: budget(B),harddrive(B,HDD)] ==>
	[nl,write('Your harddrive seems to not fit your budget. Choose from the following:'),nl,
	write(HDD),nl].

% Show the results
rule 600:
	[1: goal(show_pc),picked_cpu(CPU),picked_gpu(GPU),picked_harddrive(HDD)]
	==>	[retract(all), 	nl,write('Your configuration looks as follows:'),nl,nl,
	write('CPU: '),write(CPU),nl, write('GPU: '),write(GPU),nl, write('Harddrive: '),write(HDD),nl.
\end{lstlisting}


\end{document}