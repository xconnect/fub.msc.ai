\documentclass{article}
\usepackage[utf8]{inputenc}
\usepackage[T1]{fontenc}
\usepackage{ngerman}
 
\title{Künstliche Intelligenz\\~\\Hausaufgabe 3\\ \small{N. Lehmann, A. Zubarev}}
\date{10.05.2015}


\begin{document}

\maketitle

\section{Wahrheitstabellen}

\subsection{Teilaufgabe a)}

\begin{tabular}{|c|c|c|c|c|c|c|c|}
\hline
$p$ & $q$ & $r$ & $p \vee q \vee r$ & $r \Rightarrow (p \vee q)$ & $(q \wedge r) \Rightarrow p$ & $\neg p \vee q \vee r$ & $q \Rightarrow p \vee \neg (q \Rightarrow (p \vee r))$\\
\hline
0 & 0 & 0 & 0 & - & - & - & -\\
0 & 0 & 1 & 1 & 0 & - & - & 1\\
0 & 1 & 0 & 1 & 1 & 1 & 1 & 1\\
0 & 1 & 1 & 1 & 1 & 0 & - & -\\
1 & 0 & 0 & 1 & 1 & 1 & 0 & -\\
1 & 0 & 1 & 1 & 1 & 1 & 1 & 1\\
1 & 1 & 0 & 1 & 1 & 1 & 1 & 1\\
1 & 1 & 1 & 1 & 1 & 1 & 1 & 1\\
\hline
\end{tabular}

\subsection{Teilaufgabe b)}

\begin{tabular}{|c|c|c|c|c|c|c|}
\hline
$p$ & $q$ & $r$ & $q \vee r$ & $q \Rightarrow \neg p$ & $\neg (r \wedge p)$ & $\neg p$\\
\hline
0 & 0 & 0 & 0 & - & - & - \\
0 & 0 & 1 & 1 & 1 & 1 & 1 \\
0 & 1 & 0 & 1 & 1 & 1 & 1 \\
0 & 1 & 1 & 1 & 1 & 1 & 1 \\
1 & 0 & 0 & 0 & - & - & - \\
1 & 0 & 1 & 1 & 1 & 0 & - \\
1 & 1 & 0 & 1 & 0 & - & - \\
1 & 1 & 1 & 1 & 0 & - & - \\
\hline
\end{tabular}

\subsection{Teilaufgabe c)}

\begin{tabular}{|c|c|c|c|c|c|c|}
\hline
$p$ & $q$ & $r$ & $p \Rightarrow q$ & $q$ & $\neg (p \wedge q)$\\
\hline
0 & 0 & 0 & 1 & 0 & - \\
0 & 0 & 1 & 1 & 0 & - \\
0 & 1 & 0 & 1 & 1 & 0 \\
0 & 1 & 1 & 1 & 1 & 0 \\
1 & 0 & 0 & 0 & - & - \\
1 & 0 & 1 & 0 & - & - \\
1 & 1 & 0 & 1 & 1 & 0 \\
1 & 1 & 1 & 1 & 1 & 0 \\
\hline
\end{tabular}

\section{Alternative Repräsentation von Hornklauseln}

\subsection{Teilaufgabe a)}

\begin{tabular}{|c|c|c|c|c|c|}
\hline
$p$ & $q$ & $r$ & $c$ & $p \wedge q \wedge r$ & $(p \wedge q \wedge r) \Rightarrow c$\\
\hline
0 & 0 & 0 & 0 & 0 & 1\\
0 & 0 & 0 & 1 & 0 & 1\\
0 & 0 & 1 & 0 & 0 & 1\\
0 & 0 & 1 & 1 & 0 & 1\\
0 & 1 & 0 & 0 & 0 & 1\\
0 & 1 & 0 & 1 & 0 & 1\\
0 & 1 & 1 & 0 & 0 & 1\\
0 & 1 & 1 & 1 & 0 & 1\\
1 & 0 & 0 & 0 & 0 & 1\\
1 & 0 & 0 & 1 & 0 & 1\\
1 & 0 & 1 & 0 & 0 & 1\\
1 & 0 & 1 & 1 & 0 & 1\\
1 & 1 & 0 & 0 & 0 & 1\\
1 & 1 & 0 & 1 & 0 & 1\\
1 & 1 & 1 & 0 & 1 & 0\\
1 & 1 & 1 & 1 & 1 & 1\\
\hline
\end{tabular}

\subsection{Teilaufgabe b)}

\begin{tabular}{|c|c|c|c|c|c|c|c|}
\hline
$c$ & $p$ & $q$ & $r$ & $\neg p$ & $\neg q$ & $\neg r$ & $c \vee \neg p \vee \neg q \vee r$\\
\hline
0 & 0 & 0 & 0 & 1 & 1 & 1 & 1 \\
0 & 0 & 0 & 1 & 1 & 1 & 0 & 1 \\
0 & 0 & 1 & 0 & 1 & 0 & 1 & 1 \\
0 & 0 & 1 & 1 & 1 & 0 & 0 & 1 \\
0 & 1 & 0 & 0 & 0 & 1 & 1 & 1 \\
0 & 1 & 0 & 1 & 0 & 1 & 0 & 1 \\
0 & 1 & 1 & 0 & 0 & 0 & 1 & 1 \\
0 & 1 & 1 & 1 & 0 & 0 & 0 & 0 \\
1 & 0 & 0 & 0 & 1 & 1 & 1 & 1 \\
1 & 0 & 0 & 1 & 1 & 1 & 0 & 1 \\
1 & 0 & 1 & 0 & 1 & 0 & 1 & 1 \\
1 & 0 & 1 & 1 & 1 & 0 & 0 & 1 \\
1 & 1 & 0 & 0 & 0 & 1 & 1 & 1 \\
1 & 1 & 0 & 1 & 0 & 1 & 0 & 1 \\
1 & 1 & 1 & 0 & 0 & 0 & 1 & 1 \\
1 & 1 & 1 & 1 & 0 & 0 & 0 & 1 \\
\hline
\end{tabular}

\section{SLD-Resolution}

\subsection{Teilaufgabe a)}

\begin{itemize}
\item[\textbf{$H$}] Der Händler ist ehrlich.
\item[\textbf{$S$}] Das Saatgut ist gut.
\item[\textbf{$W_1$}] Das Wetter ist gut.
\item[\textbf{$G$}] Horst hat ausreichend Geld.
\item[\textbf{$W_2$}] Horst fährt weg.
\item[\textbf{$K$}] Es ist Kirmes.
\item[\textbf{$A$}] Borsti ist allein.
\item[\textbf{$W_3$}] Der Wolf ist satt.
\item[\textbf{$B$}] Borsti wird gefressen.
\end{itemize}

\newpage

\subsection{Teilaufgabe b)}

Annahme:\\
$K$\\
\\
Frage:\\
$\neg ((W_1 \wedge H) \Rightarrow \neg B)$\\
\\
Umformung der Frage:\\
$\neg ((W_1 \wedge H) \Rightarrow \neg B) \Leftrightarrow \neg ( \neg (W_1 \wedge H) \vee \neg B) \Leftrightarrow ((W_1 \wedge H) \wedge B)$\\
\\
Daraus folgt Menge von Hornformeln:\\
$\{ \{\neg H,S\}, \{\neg S,\neg W_1,G\}, \{\neg G,W_2\}, \{\neg W_2,\neg K,A\}, \{\neg W_2,W_3\}, \{\neg W_3,\neg B\}, \{A,\neg B\}, \{W_1\}\, \{H\}, \{B\}\}$\\
\\
SLD-Rsolution:\\
$1: \{\neg H,S\}$\\
$2: \{\neg S,\neg W_1,G\}$\\
$3: \{\neg G,W_2\}$\\
$4: \{\neg W_2,\neg K,A\}$\\
$5: \{\neg W_2,W_3\}$\\
$6: \{\neg W_3,\neg B\}$\\
$7: \{A,\neg B\}$\\
$8: \{W_1\}$\\
$9: \{H\}$\\
$10: \{B\}$\\
$11: 6+10 = \{ \neg W_3 \}$\\
$12: 7+10 = \{A \}$\\
$13: 1+9 = \{ S \}$\\
$14: 2+8 = \{ \neg S, G \}$\\
$15: 2+13 = \{ \neg W_1,G \}$\\
$16: 13+14 = \{G \}$\\
$17: 3+16 = \{ W_2 \}$
$18: 4+17 = \{ \neg K, A \}$\\
$19: 5+17 = \{ W_3 \}$\\
$20: 6+19 = \{ \neg B \}$\\
$21: 11+19 = \emptyset$ oder $22: 10+20 = \emptyset$

\subsection{Teilaufgabe c)}

$fof(1,axiom, \sim h \, \& \, s).$\\
$fof(2,axiom, \sim s \, \& \sim w1 \, \& \, g).$\\
$fof(3,axiom, \sim g \, \& \, w2).$\\
$fof(4,axiom, \sim w2 \, \& \, \sim k \, \& \, a).$\\
$fof(5,axiom, \sim w2 \, \& \, w3).$\\
$fof(6,axiom, \sim w3 \, \& \, \sim b).$\\
$fof(7,axiom, a \, \& \, \sim b).$\\
$fof(8,conjecture, w1 \, \& \, h \, \& \, b).$

\end{document}